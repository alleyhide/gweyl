\documentclass[fleqn]{jsarticle}
\usepackage{amsmath}
\usepackage{newtxtext,newtxmath}
\usepackage{theorem}
\theorembodyfont{\normalfont}
\newtheorem{theorem}{定理}
\newtheorem{definition}[theorem]{定義}
\newtheorem{problem}{問題}
%\def\languagename{hoge}
%\setbeamertemplate{theorems}[numbered]
\begin{document}

\title{ブログテスト}
\author{佐々 野寄}
\date{2020年8月15日始 \\ \today}
\maketitle

\section{20200815}

hello こんにちはああああ
さっきまで 入力できますか
gedit aaa aaaaaa


はっきり言って日本語入力するだけで、一時間くらいかかった。。。。
まずはtexをちゃんと打てるようになる環境を準備するかな

texliveをインストールしてみる

ptex2pdf -l tmp.tex
で最近はいいらし
スクリプト使う必要なし

\subsection{latexの設定}

texを使えるようになるまで一日かかるとか、ありえないよ。

texstudioをインストールして、
Option $\to$ Configure Texstudio
で、Latexを
$
ptex2pdf -l \%.tex
$
にする。

\subsection{アイデア}
\begin{itemize}
\item 恒等式から
\item 三角関数の加法定理の証明
\item $\pi > 3.05$とか
\end{itemize}


\section{20200822}

「数学は思想である」という主張は、数学の理論は長大な理論であり深謀遠慮が
垣間見えるほど深淵は思想ようだ、という意味ではない。
「数学は思想である」というのは言葉そのままである。「思想」を「考え方」
に変えるともっとよいだろうか。「数学は考え方である」。「数学」とは
何かを考えた考え方だ。これを分かるようにエッセーを書いていきたい。

学校数学から「数学は考え方」という発想にはなりにくいのかもしれない。
だから、このエッセーは価値を持ち始める可能性を持っている。
いや、そう信じる、信じたい。

私の数学人生は、もう一度再生させる。
もう、新しい定理をやる必要はない、
みんなが知っていることをつらつら書いても
誰も怒る人などいないんだ。



\section[恒等式]{恒等式の問題}

教科書に載っている何でもない問題から
何か考えられないかなぁ、というところです。

\subsection[問題]{問題提議}

まずは、問題を書いてみましょう。

\begin{problem}
\begin{equation}\label{eq:idmon}
x^3=a(x-1)^3+b(x-1)^2+c(x-1)+d
\end{equation}
が$x$に関する恒等式となるように定数$a,b,c,d$を求めよ。
\end{problem}

\subsection[定義]{恒等式の定義}

\subsection[係数比較]{係数比較法}


\section[加法定理]{三角関数の加法定理}

\subsection[教科書の方法]{教科書の証明方法}

\subsection[回転の行列]{回転の行列を使う方法}

\subsection[複素平面]{複素平面の回転移動}

\subsection[オイラーの公式]{オイラーの公式を用いる方法}

\subsection[テイラー級数]{オイラーの公式を正当化}

\subsection[微分方程式]{微分方程式による特徴付け}


\section[行列式の特徴付け]{行列式の特徴付け}


\end{document}
